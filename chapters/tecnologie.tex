In questa sezione vengono brevemente introdotte le tecnologie utilizzate.

\section{Instant Developer}
La piattaforma web su cui è stato integrato il chatbot (Active Demand) è stata sviluppata usando il framework di sviluppo \textbf{Instant Developer (In.de)} con cui Bookmark crea la maggior parte delle sue soluzioni software.

In.de è un proprio sistema di sviluppo con cui è possibile creare \textit{Rich Internet Application} e che nasce con l'obiettivo di ridurre i tempi di sviluppo, astranedo dai dettagli tecnici di basso livello, quali la scelta della specifica implementazione tecnologica sottostante, quindi disaccoppiando l'ambito applicativo da quello tecnologico.
%
Queste applicazioni sono poi automaticamente tradotte e compilate sia in linguaggio \texttt{Java} che \texttt{C\#}, in modo da funzionare su qualunque server. 

Una peculiarità di questo sistema di sviluppo è che è incentrato sulla programmazione relazionale: il codice non viene memorizzato in file di testo ma in un grafo le cui relazioni vengono tracciate dall'IDE. 
%
Questo dovrebbe sollevare, almeno in parte, stando agli intenti degli ideatori di In.de, il programmatore dal dover riadattare il software a fronte di cambianti nelle specifiche di progetto.

\section{Instant Support Editor}
\textit{Instant Support Editor} è, invece, un'applicazione web per la gestione e configurazione di tutto ciò che concerne il cuore del chatbot: dalla scrittura delle definizioni, al concepimento dei flussi.
