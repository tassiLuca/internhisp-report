I \textit{chatbot} sono software progettati per simulare una conversazione umana con interazioni sia testuali che vocali.
%
Negli ultimi anni, grazie agli importanti progressi nel campo dell'intelligenza artificiale e del \textit{machine learning}, questi assistenti virtuali si sono diffusi in una molteplicità di servizi, dall'ordinazione a domicilio all'assistenza clienti, la quale viene svolta da persone fisiche sempre più raramente.

Bookmark s.r.l. è una società informatica che sviluppa software gestionali web e, come tutte le aziende del settore, deve, per contratto, garantire la necessaria assistenza ai propri clienti. 
%
Tuttavia, una considerevole quantità di richieste di assistenza potrebbe essere smaltita "a monte" mediante l'uso di un assistente virtuale che, riuscendo ad interpretare la domanda dell'utente, è in grado di guidarlo e fornire una prima rapida forma di assistenza.

\`E proprio in questo contesto che si innesta l'attività del tirocinio: lo scopo è stato quello di integrare un chatbot all'interno di \textit{Active Demand}, una piattaforma gestionale web che integra i principali processi commerciali: contratti, politiche di \textit{pricing}, gestione magazzino e promozioni.
%
In particolare, l'obiettivo di Bookmark era di riuscire ad implementare un servizio che, una vola a regime, fosse in grado, da una parte di fornire un supporto più celere possibile all'utente, dall'altra alleggerire il carico sul personale addetto all'assistenza clienti.

\begin{figure}
    \centering{}
    \includegraphics*[width=\textwidth]{./img/ad.png}
    \caption{Active Demand}
    \label{img:ad}
\end{figure}

Le linee guida generali richieste da Bookmark che hanno guidato la configurazione del \textit{chatbot} all'interno dell'applicativo si possono riassumere nei seguenti punti:
\begin{itemize}
    \item cercare di configurare al meglio il chatbot in modo tale da ridurre più possibile fonti errore e/o \textit{misunderstanding} da parte del \textit{chatbot}.
    \item laddove ripetutamente il chatbot non sia in grado di capire la domanda (e quindi dare una risposta), passare il controllo ad un operatore;
    \item nel caso in cui un operatore non sia disponibile dare la possibilità di aprire un nuovo \textit{ticket} di supporto. I \textit{ticket} di supporto sono, infatti, la modalità contrattualmente garantita da Bookmark per l'assistenza clienti.
\end{itemize}

