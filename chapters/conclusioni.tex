\begin{figure}[h]
    \centering{}
    \includegraphics*[width=\textwidth]{./img/ad-login.png}
    \caption{Il chatbot installato sulla piattaforma Active Demand}
    \label{fig:ad-login}
\end{figure}

Il \textit{chatbot} è stato integrato all'interno di Active Demand e sono state configurate le principali entità e \textit{intent}, fondamentali per il riconoscimento delle frasi dell'utente.

\`E quindi stato predisposto per, sia passare il controllo ad un operatore disponibile, sia creare un nuovo \textit{ticket} di supporto.

L'ultimo passo, rimasto incompleto, è la configurazione di tutti i flussi necessari: questo compito dovrebbe essere svolto a stretto contatto con l'utenza \textit{target} dell'applicazione in modo da configurare in maniera precisa le risposte da fornire ai principali problemi che sorgono nell'uso quotidiano dell'applicazione.

Concludendo, questa esperienza mi ha introdotto al mondo dei \textit{chatbot} e degli assistenti virtuali. 

Mi sarei aspettato di toccare maggiormente con mano e approfondire gli aspetti di programmazione e di processi di sviluppo che Bookmark utilizza, mentre l'attività svolta è stata prevalentemente di configurazione del \textit{chatbot} in loro possesso.
