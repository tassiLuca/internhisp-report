\documentclass[a4paper,twoside,11pt]{report}

\usepackage{alltt, fancyvrb, url}
\usepackage{graphicx}
\usepackage{subfig}
\usepackage[utf8]{inputenc}
\usepackage{float}
\usepackage{hyperref}
\usepackage{import}
% Questo commentalo se vuoi scrivere in inglese.
\usepackage[italian]{babel}
\usepackage[italian]{cleveref}

\title{
	Integrazione di un chatbot ad "oracolo" in un software gestionale in ambito  retail \\
	\large Relazione finale di tirocinio c/o Bookmark s.r.l.
}

\author{
	\textbf{Luca Tassinari} \\\\
	Tutor Accademico: Prof. Mirko Viroli \\
	Tutor aziendale: Antonio Rughi
}

\date{02/05/2022 - 10/06/2022}

\begin{document}

\maketitle

\begin{abstract}
I chatbot, assistenti virtuali, sono tra gli strumenti che, grazie ai progressi nel campo dell'intelligenza artificiale, stanno riscuotendo notevole successo per aiutare e guidare l'utente nell'utilizzo di nuove tecnologie.

La Bookmark s.r.l. è una società informatica che sviluppa software gestionali web e, come tutte le aziende del settore, deve, per contratto, garantire la necessaria assistenza ai propri clienti. 
%
Tuttavia, una considerevole quantità di richieste di assistenza potrebbe essere smaltita "a monte" mediante l'uso di un assistente virtuale che, riuscendo ad interpretare la domanda dell'utente, è in grado di guidarlo e fornire una prima rapida forma di assistenza.

Scopo di questo tirocinio era quindi quello di integrare un chatbot all'interno di una piattaforma gestionale già esistente e funzionante finalizzato ad aziende di distribuzione e retail.
\end{abstract}

\tableofcontents

\chapter[Introduzione]{Introduzione}
\import{chapters/}{intro.tex}

\chapter[Tecnologie]{Tecnologie impiegate}
\import{chapters/}{tecnologie}

\chapter[Attività]{Attività svolte}

\chapter[Conclusioni]{Conclusioni e commenti finali}

\bibliographystyle{alpha}
\bibliography{report}

\end{document}
